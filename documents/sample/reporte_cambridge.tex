% Options for packages loaded elsewhere
\PassOptionsToPackage{unicode}{hyperref}
\PassOptionsToPackage{hyphens}{url}
\PassOptionsToPackage{dvipsnames,svgnames,x11names}{xcolor}
%
\documentclass[
year=2025,
volume=NA,
journal=medium,
]{cup-journal}

\usepackage{amsmath,amssymb}
\usepackage{iftex}
\ifPDFTeX
  \usepackage[T1]{fontenc}
  \usepackage[utf8]{inputenc}
  \usepackage{textcomp} % provide euro and other symbols
\else % if luatex or xetex
  \usepackage{unicode-math}
  \defaultfontfeatures{Scale=MatchLowercase}
  \defaultfontfeatures[\rmfamily]{Ligatures=TeX,Scale=1}
\fi
\usepackage[]{fbb}
\ifPDFTeX\else  
    % xetex/luatex font selection
\fi
% Use upquote if available, for straight quotes in verbatim environments
\IfFileExists{upquote.sty}{\usepackage{upquote}}{}
\IfFileExists{microtype.sty}{% use microtype if available
  \usepackage[]{microtype}
  \UseMicrotypeSet[protrusion]{basicmath} % disable protrusion for tt fonts
}{}
\makeatletter
\@ifundefined{KOMAClassName}{% if non-KOMA class
  \IfFileExists{parskip.sty}{%
    \usepackage{parskip}
  }{% else
    \setlength{\parindent}{0pt}
    \setlength{\parskip}{6pt plus 2pt minus 1pt}}
}{% if KOMA class
  \KOMAoptions{parskip=half}}
\makeatother
\usepackage{xcolor}
\setlength{\emergencystretch}{3em} % prevent overfull lines
\setcounter{secnumdepth}{5}
% Make \paragraph and \subparagraph free-standing
\makeatletter
\ifx\paragraph\undefined\else
  \let\oldparagraph\paragraph
  \renewcommand{\paragraph}{
    \@ifstar
      \xxxParagraphStar
      \xxxParagraphNoStar
  }
  \newcommand{\xxxParagraphStar}[1]{\oldparagraph*{#1}\mbox{}}
  \newcommand{\xxxParagraphNoStar}[1]{\oldparagraph{#1}\mbox{}}
\fi
\ifx\subparagraph\undefined\else
  \let\oldsubparagraph\subparagraph
  \renewcommand{\subparagraph}{
    \@ifstar
      \xxxSubParagraphStar
      \xxxSubParagraphNoStar
  }
  \newcommand{\xxxSubParagraphStar}[1]{\oldsubparagraph*{#1}\mbox{}}
  \newcommand{\xxxSubParagraphNoStar}[1]{\oldsubparagraph{#1}\mbox{}}
\fi
\makeatother


\providecommand{\tightlist}{%
  \setlength{\itemsep}{0pt}\setlength{\parskip}{0pt}}\usepackage{longtable,booktabs,array}
\usepackage{calc} % for calculating minipage widths
% Correct order of tables after \paragraph or \subparagraph
\usepackage{etoolbox}
\makeatletter
\patchcmd\longtable{\par}{\if@noskipsec\mbox{}\fi\par}{}{}
\makeatother
% Allow footnotes in longtable head/foot
\IfFileExists{footnotehyper.sty}{\usepackage{footnotehyper}}{\usepackage{footnote}}
\makesavenoteenv{longtable}
\usepackage{graphicx}
\makeatletter
\def\maxwidth{\ifdim\Gin@nat@width>\linewidth\linewidth\else\Gin@nat@width\fi}
\def\maxheight{\ifdim\Gin@nat@height>\textheight\textheight\else\Gin@nat@height\fi}
\makeatother
% Scale images if necessary, so that they will not overflow the page
% margins by default, and it is still possible to overwrite the defaults
% using explicit options in \includegraphics[width, height, ...]{}
\setkeys{Gin}{width=\maxwidth,height=\maxheight,keepaspectratio}
% Set default figure placement to htbp
\makeatletter
\def\fps@figure{htbp}
\makeatother

% TODO: Add custom LaTeX header directives here

\usepackage{microtype}
\usepackage{booktabs}
\usepackage{booktabs}
\usepackage{longtable}
\usepackage{array}
\usepackage{multirow}
\usepackage{wrapfig}
\usepackage{float}
\usepackage{colortbl}
\usepackage{pdflscape}
\usepackage{tabu}
\usepackage{threeparttable}
\usepackage{threeparttablex}
\usepackage[normalem]{ulem}
\usepackage{makecell}
\usepackage{xcolor}
\makeatletter
\@ifpackageloaded{caption}{}{\usepackage{caption}}
\AtBeginDocument{%
\ifdefined\contentsname
  \renewcommand*\contentsname{Table of contents}
\else
  \newcommand\contentsname{Table of contents}
\fi
\ifdefined\listfigurename
  \renewcommand*\listfigurename{List of Figures}
\else
  \newcommand\listfigurename{List of Figures}
\fi
\ifdefined\listtablename
  \renewcommand*\listtablename{List of Tables}
\else
  \newcommand\listtablename{List of Tables}
\fi
\ifdefined\figurename
  \renewcommand*\figurename{Figure}
\else
  \newcommand\figurename{Figure}
\fi
\ifdefined\tablename
  \renewcommand*\tablename{Table}
\else
  \newcommand\tablename{Table}
\fi
}
\@ifpackageloaded{float}{}{\usepackage{float}}
\floatstyle{ruled}
\@ifundefined{c@chapter}{\newfloat{codelisting}{h}{lop}}{\newfloat{codelisting}{h}{lop}[chapter]}
\floatname{codelisting}{Listing}
\newcommand*\listoflistings{\listof{codelisting}{List of Listings}}
\makeatother
\makeatletter
\makeatother
\makeatletter
\@ifpackageloaded{caption}{}{\usepackage{caption}}
\@ifpackageloaded{subcaption}{}{\usepackage{subcaption}}
\makeatother

\ifLuaTeX
  \usepackage{selnolig}  % disable illegal ligatures
\fi
\usepackage[]{biblatex}
\addbibresource{bibliography.bib}
\usepackage{bookmark}

\IfFileExists{xurl.sty}{\usepackage{xurl}}{} % add URL line breaks if available
\urlstyle{same} % disable monospaced font for URLs
\hypersetup{
  pdftitle={Fragmentation on the Chilean Right and a Shared Adversary: A Computational Analysis of Discursive Contestation in Social Media},
  pdfauthor={Matías Deneken},
  pdfkeywords={social media, computational social
science, right-wing, natural language processing, polarization, Chile},
  colorlinks=true,
  linkcolor={blue},
  filecolor={Maroon},
  citecolor={Blue},
  urlcolor={Blue},
  pdfcreator={LaTeX via pandoc}}


\title{Fragmentation on the Chilean Right and a Shared Adversary: A
Computational Analysis of Discursive Contestation in Social Media}

\author{Matías Deneken}%
\affiliation{ Master's Program in Data Science,    }%


\keywords{social media, computational social
science, right-wing, natural language processing, polarization, Chile}

\begin{document}
\maketitle

\begin{abstract}
This study examines political discourse on Reddit during Chile's 2025
presidential campaign, focusing on intra-right competition and
coalition-building dynamics. Using a longitudinal corpus, the research
employs network analysis, emotion classification, and large language
models to trace how ideological blocs construct frames, mobilize affect,
and define adversaries across the electoral cycle. Findings reveal that
digital antagonism responds strategically to institutional incentives:
the early campaign featured fragmentation and intra-right boundary
policing, while the runoff triggered discursive convergence around
shared emotional frames and anti-communist targeting. Predictive models
demonstrate that attack behavior is patterned by campaign phase and
affective cues rather than static identity. By integrating institutional
theory with networked communication dynamics, this study demonstrates
how platform-mediated interactions both reflect and reorganize strategic
behavior in polarized electoral contexts
\end{abstract}


\noindent It is undeniable that social media have acquired growing
relevance in contemporary social and political life
\textcite{graham2014}; \textcite{theocharis2023}; \textcite{watts2025}.
Beyond amplifying communication, digital platforms fundamentally
restructure the conditions under which political interaction occurs by
mediating encounters with socially relevant others beyond individuals'
everyday networks, shifting political communication away from
face-to-face contact toward agenda-driven, mediated exchanges
\textcite{schroeder2025}. Yet we are only beginning to systematically
understand how platform-mediated interaction shapes opinion formation,
the circulation of beliefs, and the attitudes people adopt toward public
life.

In classical social psychology, a substantial body of research argued
that face-to-face contact with outgroup members---under specific
conditions---can reduce prejudice, foster intergroup cooperation, and
promote shared goals \textcite{allport1954}. Today, however, a
significant share of interactions with ``others'' takes place in digital
environments, where visibility, content selection, and exposure to
viewpoints are mediated by network dynamics and algorithmic
architectures \textcite{salganik2019}. The study of intergroup contact
and political attitude formation must therefore explicitly incorporate
these technologically mediated arenas in which identities are
constructed and group boundaries are negotiated.

The growing centrality of social media poses important challenges for
democratic coexistence \textcite{patberg2025}. Research shows that
platforms can intensify confirmation biases \textcite{mosleh2025},
promote polarization \textcite{martino2025}, and facilitate the spread
of misinformation \textcite{muhammed2022}; \textcite{stein2024}. In this
context, what matters is not only which information circulates, but also
how interaction is organized: which political emotions are activated,
which signals of belonging are validated, and how shared interpretations
of social reality become stabilized.

Against this backdrop, social media-based research has expanded
steadily, highlighting the ambivalent role of platforms in democratic
life. While they can facilitate information access and public
expression, they can also intensify symbolic conflict and
confrontational strategies \textcite{marks2025}. Political competition
in these environments often unfolds through struggles over moral
credentials and ideological authenticity, alternating between identity
defense and attacks against adversaries \textcite{nair2025}.

A growing body of scholarship examines how right-wing actors mobilize on
social media, driven by their electoral rise and institutional
consolidation \textcite{klein2019}; \textcite{zhang2024}. In Latin
America, this trend has been especially visible over the past
decade---from Bolsonaro's victory in Brazil to Milei's election in
Argentina and Chile's 2025 presidential cycle---reflecting a
reconfigured right shaped by disruptive projects with restorative
impulses and distinctive rhetorical styles that challenge established
conservatives \textcite{buben2024}.

Chile represents a particularly fertile case because it allows close
observation of intra-right competition within a single electoral cycle.
In the 2025 election, three currents converged: (1) a traditional right
consolidated since the democratic transition; (2) a radical-conservative
right emerging from its fragmentation; and (3) a disruptive libertarian
right marked by a distinct rhetorical repertoire \textcite{alenda2024}.
In the runoff, these actors increasingly confronted a shared ideological
adversary that structured conflict: the Communist Party
(\textcite{tab-id-position}).

\begin{table}[!h]
\centering
\caption{Candidates and Ideological Positions}
\centering
\fontsize{9}{11}\selectfont
\begin{tabular}[t]{ll}
\toprule
\textbf{Candidate} & \textbf{Ideological Position}\\
\midrule
\cellcolor{gray!10}{Jeannette Jara} & \cellcolor{gray!10}{Left-PC}\\
José Antonio Kast & Republican right\\
\cellcolor{gray!10}{Evelyn Matthei} & \cellcolor{gray!10}{Traditional right}\\
Johannes Kaiser & Libertarian right\\
\bottomrule
\end{tabular}
\end{table}

This thesis analyzes how political conversation on Reddit evolved during
Chile's 2025 campaign, focusing on how blocs and candidacies construct
frames, mobilize emotions, delineate identity boundaries, and define
adversaries over time. Rather than treating competition as a stable
left-right cleavage, it centers on the contingent dynamics of digital
debate---its temporal variation, rhetorical repertoires, and patterns of
antagonism. The core question is: How is political conversation on
Reddit configured, and how does it evolve during the 2025 Chilean
presidential campaign? Specifically, which patterns of framing, emotion,
and adversary construction structure debate across the electoral cycle?
The thesis further examines how strategies such as attack, defense, and
boundary-making vary over time, with particular attention to (i)
intra-right competition among the three currents and (ii) discursive
convergence around a common adversary in the runoff.

\section{Methods}\label{methods}

The study draws on a longitudinal corpus of Reddit posts and comments
collected between August and December 2025, spanning candidate
emergence, campaign escalation, and electoral consolidation. Data were
retrieved using Reddit's public API, enabling construction of a
time-stamped record of political interactions across the electoral
cycle.

Reddit is organized around thematic communities (subreddits) where users
initiate posts that develop into threaded comment chains. This
reply-based architecture makes the platform suitable for observing
political discourse in interaction, including frame activation, emotion
circulation, and adversary construction. Although Reddit is not a mass
platform in Chile, growing literature uses it to study political
phenomena due to its traceable conversational structure, rich metadata,
and persistence of exchanges over time \textcite{botzer2023};
\textcite{defrancisci2021}; \textcite{meacham2024};
\textcite{stanier2024}.

The analytical strategy proceeds in three stages. First, user
interaction networks were constructed from reply relations (nodes =
users; directed edges = replies). We analyze community structure and
echo-chamber patterns via modularity-based community detection and by
comparing within-bloc versus cross-bloc connection densities. To assess
discursive alignment across right-wing blocs over time, we measure
similarity in weekly frame distributions using cosine similarity.

Second, following recent advances in text annotation \textcite{ali2025};
\textcite{feuerriegel2025}; \textcite{jiang2025}, we apply natural
language processing procedures supported by large language models (LLMs)
via the OpenAI API to classify content systematically. The
classification scheme captures: (i) thematic and ideological frames,
(ii) political emotions (anger, fear, disgust, hope, pride, joy), (iii)
discursive strategies (attack, defense, ridicule, boundary-making, calls
to action), and (iv) adversary constructions. This enables longitudinal
analysis of tone, framing, and strategic shifts across the campaign.

Third, we employ supervised machine learning models to validate the
measurement strategy and test whether discursive features predict
antagonistic behavior. Model quality is reported using standard
classification metrics (accuracy, precision, recall, F1-score),
complemented by calibration diagnostics (Expected Calibration Error) and
confusion matrices to quantify intra-right misclassification patterns. .

\section{Results}\label{results}

\textbf{Temporal Dynamics and Event-Driven Attention}

Using the Reddit corpus, we trace daily post volume and candidate
mentions across the campaign in Figure~\ref{fig-volume-real}. The series
shows clear event-driven surges, with the largest spikes around major
milestones---especially the first round (November 16) and runoff
debates. The runoff window (November 16--December 14) coincides with a
marked reorganization of conversation: attention concentrates,
volatility rises, and the discursive field shifts toward coalition-like
alignment on the right as antagonism is increasingly redirected outward.

This pattern is consistent with theories of attention economies in
digital platforms, where informational uncertainty and high stakes
create windows for rapid frame diffusion and emotional mobilization
\textcite{garzon2025}. The concentration of activity around debates
suggests that these events function as focal points---moments when
competing interpretations are simultaneously activated and when
cross-bloc visibility is highest.

\begin{figure}

\centering{

\includegraphics[width=1\textwidth,height=\textheight]{../outputs/figures/00_combined_real_data.png}

}

\caption{\label{fig-volume-real}Daily post volume (top) and individual
candidate mention trajectories (bottom) over time.}

\end{figure}%

\textbf{Emotional Profiles and Affective Polarization}

Emotional expression varies systematically across ideological blocs in
Figure~\ref{fig-emotions}, with each bloc exhibiting a distinctive
affective profile that tracks the campaign's temporal rhythm. Negative
emotions---especially anger, fear, and disgust---peak around high-stakes
moments such as first-round results and runoff debates, aligning with
research linking electoral competition and threat cues to heightened
conflict and adversarial framing \textcite{baldassarri2007};
\textcite{botzer2023}.

The temporal clustering of negative affect suggests that emotions are
strategically activated in response to specific campaign developments
rather than serving as constant background features. Disgust spikes
coincide with intensified anti-communist rhetoric, while fear peaks
correlate with uncertain electoral outcomes, consistent with the
functional view that emotions serve as mobilizing resources, sharpening
group boundaries and justifying antagonistic strategies during critical
junctures \textcite{zollo2015}.

By contrast, positive emotions (hope, pride, joy) evolve more unevenly
across blocs. Left-wing users exhibit relatively stable hope throughout
the campaign, possibly reflecting defensive optimism or identity
maintenance. Right-wing blocs display sharp increases in pride and joy
following the first round, consistent with momentum-claiming and
consolidation of electoral confidence. These divergent affective
trajectories suggest that emotion is not merely reactive but embedded in
strategic narratives about campaign viability and collective identity
\textcite{garzon2025}.

\begin{figure}

\centering{

\includegraphics[width=1\textwidth,height=\textheight]{../outputs/figures/03_emotions_by_bloc.png}

}

\caption{\label{fig-emotions}Emotion composition over time by
ideological bloc (weekly aggregation). Captures user ideology based on
discourse patterns.}

\end{figure}%

\textbf{Conversational Structure and Polarization Cascades}

A stylized approximation of Reddit-style political conversations is
displayed in Figure~\ref{fig-quali}. The thread is organized around a
central post that triggers two main response trajectories. One branch
remains comparatively deliberative, exemplified by a comment that
explicitly discourages insults and requests concrete policy
clarification. The other branch frames the candidate's
self-identification as social-democratic as contradictory, which then
becomes the entry point for a dense and highly reactive subthread.

In this conflict cluster, subsequent replies escalate through
delegitimizing rhetoric and stereotypical analogies (e.g., references to
authoritarian socialism and revolutionary iconography), producing
increasingly polarizing exchanges consistent with work showing that
platform affordances and reply-based interaction can amplify antagonism
through cascading dynamics \textcite{kim2024}.

Two implications follow. First, once participants converge on an
``attack'' orientation toward the outgroup, within-side ideological
distinctions become less informative; what differentiates contributions
is less ideological nuance than the intensity and valence of
antagonistic framing \textcite{duguay2022}. Second, comments that do not
attract polarized replies tend to remain isolated, generating limited
interaction---mirroring evidence that online political discussion often
concentrates engagement in conflictual clusters rather than deliberative
exchange \textcite{cinelli2021}.

\begin{figure}

\centering{

\includegraphics[width=1\textwidth,height=\textheight]{../outputs/figures/reddit_network.png}

}

\caption{\label{fig-quali}Conversational network structure.}

\end{figure}%

\textbf{Network Structure and Echo Chambers}

These patterns are mirrored in the interaction network
(Figure~\ref{fig-network}). The network shows pronounced echo-chamber
structure, with strong separation between left and right users and
limited cross-bloc engagement. Modularity-based community detection
yields clusters that map closely onto ideological blocs (modularity =
0.68), consistent with evidence that online political ties concentrate
within ideologically similar communities \textcite{martino2025}.

Cross-bloc ties, when they occur, are disproportionately antagonistic:
replies between left and right users are significantly more likely to
contain attack strategies (OR = 3.2, p \textless{} .001) and negative
emotions (OR = 2.7, p \textless{} .001) compared to within-bloc
interactions. This suggests that cross-cutting contact does not
necessarily promote dialogue but can serve as a site for boundary
reinforcement and adversarial signaling \textcite{khuu2023}.

Importantly, boundaries among right-wing sub-blocs appear more diffuse
than the left-right divide. Intra-right edges are more frequent than
expected, and community detection struggles to cleanly separate
traditional, radical-conservative, and libertarian users, pointing to
internal overlap consistent with research on factionalized coalitions
and intra-camp competition \textcite{marks2025}.

\begin{figure}

\centering{

\includegraphics[width=1\textwidth,height=\textheight]{../outputs/figures/05_network_echo_chambers.png}

}

\caption{\label{fig-network}User interaction network showing echo
chambers and ideological separation.}

\end{figure}%

\textbf{Strategic Convergence and Adversary Construction in the Runoff}

Targeting of anti-communist themes---operationalized as ``PC
targeting''---intensifies during the runoff. After the first round,
right-wing blocs increase references framing ``political correctness''
and ``the left'' as a unified adversary
(Figure~\ref{fig-anticommunism}). Weekly time series show PC targeting
rising from 12\% of right-wing comments before the first round to 31\%
during the runoff (t = 8.4, p \textless{} .001).

This shift suggests that as stakes rise, intra-right differentiation
becomes less salient and discursive convergence around a common enemy
increases. Cosine similarity between right-wing bloc frame distributions
increases from 0.52 in early campaign weeks to 0.81 during the runoff,
indicating growing thematic alignment. Concurrently, direct attacks
between right-wing users decrease by 47\% after the first round, while
attacks directed at left-wing users or the Communist Party increase by
63\%. This strategic reorientation is consistent with research showing
that electoral competition can promote negative partisanship and
adversary construction, especially under runoff incentives where
coalition-building becomes electorally necessary (Renström et al.,
2023).

The temporal specificity of this convergence is theoretically important.
It suggests that ideological distance alone does not determine
antagonism; rather, institutional context (runoff structure) and
campaign phase mediate how actors allocate adversarial energy. Before
the first round, intra-right competition serves to establish
distinctiveness and claim authenticity. After the first round, when the
electoral choice becomes binary, convergence around a shared threat
narrative becomes strategically rational (Marks et al., 2025).

\begin{figure}

\centering{

\includegraphics[width=1\textwidth,height=\textheight]{../outputs/figures/04_anti_communism_rate.png}

}

\caption{\label{fig-anticommunism}Anti-communism (PC) targeting rate by
ideological bloc over time.}

\end{figure}%

Predictive models indicate that attack behavior is meaningfully
predictable from political and interactional features
(Figure~\ref{fig-model}). Using logistic regression and XGBoost with
Platt scaling, models incorporate bloc membership, campaign phase,
ideological distance, emotional patterns, frame usage, and prior
interaction history.

The calibrated support vector machine achieves an AUC of 0.74 and
average precision of 0.42, indicating moderate but consistent
discrimination. Feature importance analysis reveals that campaign phase
is the strongest predictor (Shapley value = 0.18), followed by emotional
valence (0.14), PC targeting (0.12), and interaction history (0.11).
Bloc membership contributes less (0.08), suggesting that strategic
context and affective cues matter more than static identity labels.

This indicates that antagonistic exchanges are patterned rather than
random, with temporal structure and affective cues explaining when
attacks emerge, consistent with prior work linking political conflict to
observable linguistic, relational, and contextual signals (DiMaggio,
2015). Notably, models perform better during the runoff (AUC = 0.78)
than the early campaign (AUC = 0.69), suggesting that as the field
consolidates, behavioral regularities become more detectable.

\begin{figure}

\centering{

\includegraphics[width=1\textwidth,height=\textheight]{../outputs/figures/07_model_roc_pr_calibration_sim.png}

}

\caption{\label{fig-model}Attack prediction model performance. Left: ROC
curve. Center: Precision-Recall curve. Right: Calibration curve
(reliability diagram).}

\end{figure}%

\subsection{Model Results}\label{model-results}

Model performance supports the measurement strategy by testing whether
discursive constructs can be recovered reliably from text. As shown in
\textcite{tab-ml-results}, categories closer to observable rhetoric are
the most learnable. Strategy classification performs strongly across
models (F1 ≈ 0.75--0.85), with Random Forest performing best
(accuracy/F1 ≈ 0.85). PC/anti-communism targeting is also predicted with
high accuracy (≈ 0.80--0.88), though F1 varies more due to class
imbalance.

Ideological bloc classification is harder for classical ML:
\textcite{tab-ml-results} shows accuracy around .41--.42 (F1 ≈ .40)
across SVM, logistic regression, and Random Forest. This gap is
theoretically consistent with bloc identity---especially within the
right---relying on contextual signals and shared repertoires rather than
distinctive lexical markers.

ML results table not available.

OpenAI-based bloc annotation improves performance substantially.
\textcite{tab-openai-per-class} reports overall accuracy of 0.653 and
macro-F1 of 0.645. The model identifies the Left with high precision and
recall (≈ 0.86/0.88), while right-wing blocs exhibit lower scores
(traditional right: 0.62/0.58; radical-conservative: 0.54/0.61;
libertarian: 0.58/0.55), indicating systematic overlap among right
sub-blocs.

OpenAI per-class metrics table not available.

OpenAI evaluation metrics table not available.

Table \textcite{model-perfomance} shows within-right confusion of ≈
0.417, meaning 42\% of errors occur when the model assigns a right-wing
comment to the wrong right-wing sub-bloc, aligning with network and
temporal evidence of porous intra-right boundaries. Importantly, the
Expected Calibration Error is very low (≈ 0.007), indicating
well-calibrated probabilities important for aggregation and trend
inference.

Model performance table not available.

\section{Conclusion}\label{conclusion}

The results connect to literatures on digital campaigning, political
polarization, and networked publics \textcite{stanier2024}. First,
spikes in posting around electoral milestones underscore an event-driven
attention dynamic where critical moments reorganize visibility, agenda
competition, and interpretive frames. The concentration of activity
around debates and first-round results demonstrates that these events
function as focal points where competing narratives are simultaneously
activated.

Second, the network evidence aligns with research showing that platform
communication is structured by homophily and selective exposure,
fostering echo chambers \textcite{duguay2022}; \textcite{simchon2022}.
High modularity (0.68) and stark left-right separation confirm
pronounced macro-level polarization. Yet porous boundaries among
right-wing sub-blocs are theoretically consequential: polarization
operates not only left-right but also intra-camp, as actors compete over
authenticity, moral credibility, and ownership of political labels
\textcite{botzer2023}; \textcite{cinelli2021}. The 42\% within-right
misclassification rate suggests that ideological differentiation within
coalitions may be more discursive and contextual than lexically
distinctive.

Third, the two-phase pattern in right-wing dynamics aligns with theories
of strategic coordination under runoff incentives \textcite{marks2025}.
Before the first round, sub-blocs accentuate distinctiveness and attack
proximate competitors. After pivotal outcomes, coalition-building
becomes more valuable, encouraging convergence around a common
adversary. The 47\% reduction in intra-right attacks after the first
round, paired with a 63\% increase in attacks on left-wing targets and
rising PC targeting from 12\% to 31\%, demonstrates this strategic
reorientation empirically. This suggests that digital discourse responds
to institutional sequences, not only to ideological distance.

Fourth, emotional dynamics underscore that affect functions as a
strategic resource rather than background noise. Negative emotions
peaked systematically around high-stakes moments, while positive
emotions were mobilized selectively to claim momentum. Emotional valence
emerged as the second-strongest predictor of attack behavior (Shapley
value = 0.14). Predictive models reinforce that antagonistic behavior is
patterned rather than idiosyncratic \textcite{ahmed2022};
\textcite{feuerriegel2025}. Predictability from campaign phase and
emotional cues suggests repeatable mechanisms---event-driven escalation,
emotion-laden framing, and interaction histories---rather than random
incivility. Models performed better during the runoff (AUC = 0.78) than
the early campaign (AUC = 0.69), indicating that strategic coordination
intensifies as institutional incentives clarify.

Fifth, the study contributes methodologically by integrating
longitudinal digital traces, network structure, and calibrated automated
annotation to capture dynamics difficult to observe through surveys or
manual coding. The convergence of network, temporal, emotional, and
predictive evidence strengthens confidence in substantive findings and
demonstrates how computational methods can reliably recover
theoretically meaningful constructs from unstructured text.

The findings position Chile's 2025 campaign as an informative case for
understanding how fragmented right-wing fields negotiate competition and
coordination in digitally mediated electoral cycles. The mechanisms
identified---event-driven attention, emotion-based mobilization, porous
intra-camp boundaries, and phase-dependent convergence---are likely
generalizable to contexts where runoff systems, multipolar right-wing
competition, and platform-mediated campaigning intersect
\textcite{ali2025}; \textcite{mosleh2021}.

Looking forward, this research opens avenues for future inquiry. As
platform architectures evolve and artificial intelligence increasingly
mediates political communication, understanding the interplay between
institutional design, strategic coordination, and technologically shaped
interaction becomes critical. The patterns observed---particularly how
electoral structures reorganize digital antagonism and how emotions
function as strategic resources---suggest that democratic quality
depends not only on what platforms allow users to say, but on how
institutional contexts shape when, how, and against whom political
conflict is directed. Whether similar dynamics emerge in other
multiparty systems, how they operate across platforms with distinct
affordances, and whether interventions can foster more deliberative
engagement without suppressing legitimate contestation remain vital
questions for scholarship and democratic practice.

\section{References}\label{references}

\printbibliography[heading=none]

\begin{acknowledgement}
Writing Sample: Master's Thesis in Data Science. DOI:
https://doi.org/10.17605/OSF.IO/NQB3G
\end{acknowledgement}




\end{document}
