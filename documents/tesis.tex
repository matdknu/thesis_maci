% Options for packages loaded elsewhere
\PassOptionsToPackage{unicode}{hyperref}
\PassOptionsToPackage{hyphens}{url}
\PassOptionsToPackage{dvipsnames,svgnames,x11names}{xcolor}
%
\documentclass[
  12pt,
]{book}

\usepackage{amsmath,amssymb}
\usepackage{setspace}
\usepackage{iftex}
\ifPDFTeX
  \usepackage[T1]{fontenc}
  \usepackage[utf8]{inputenc}
  \usepackage{textcomp} % provide euro and other symbols
\else % if luatex or xetex
  \usepackage{unicode-math}
  \defaultfontfeatures{Scale=MatchLowercase}
  \defaultfontfeatures[\rmfamily]{Ligatures=TeX,Scale=1}
\fi
\usepackage{lmodern}
\ifPDFTeX\else  
    % xetex/luatex font selection
\fi
% Use upquote if available, for straight quotes in verbatim environments
\IfFileExists{upquote.sty}{\usepackage{upquote}}{}
\IfFileExists{microtype.sty}{% use microtype if available
  \usepackage[]{microtype}
  \UseMicrotypeSet[protrusion]{basicmath} % disable protrusion for tt fonts
}{}
\makeatletter
\@ifundefined{KOMAClassName}{% if non-KOMA class
  \IfFileExists{parskip.sty}{%
    \usepackage{parskip}
  }{% else
    \setlength{\parindent}{0pt}
    \setlength{\parskip}{6pt plus 2pt minus 1pt}}
}{% if KOMA class
  \KOMAoptions{parskip=half}}
\makeatother
\usepackage{xcolor}
\usepackage[top=30mm,bottom=30mm,left=25mm,right=25mm,heightrounded]{geometry}
\setlength{\emergencystretch}{3em} % prevent overfull lines
\setcounter{secnumdepth}{5}
% Make \paragraph and \subparagraph free-standing
\makeatletter
\ifx\paragraph\undefined\else
  \let\oldparagraph\paragraph
  \renewcommand{\paragraph}{
    \@ifstar
      \xxxParagraphStar
      \xxxParagraphNoStar
  }
  \newcommand{\xxxParagraphStar}[1]{\oldparagraph*{#1}\mbox{}}
  \newcommand{\xxxParagraphNoStar}[1]{\oldparagraph{#1}\mbox{}}
\fi
\ifx\subparagraph\undefined\else
  \let\oldsubparagraph\subparagraph
  \renewcommand{\subparagraph}{
    \@ifstar
      \xxxSubParagraphStar
      \xxxSubParagraphNoStar
  }
  \newcommand{\xxxSubParagraphStar}[1]{\oldsubparagraph*{#1}\mbox{}}
  \newcommand{\xxxSubParagraphNoStar}[1]{\oldsubparagraph{#1}\mbox{}}
\fi
\makeatother


\providecommand{\tightlist}{%
  \setlength{\itemsep}{0pt}\setlength{\parskip}{0pt}}\usepackage{longtable,booktabs,array}
\usepackage{calc} % for calculating minipage widths
% Correct order of tables after \paragraph or \subparagraph
\usepackage{etoolbox}
\makeatletter
\patchcmd\longtable{\par}{\if@noskipsec\mbox{}\fi\par}{}{}
\makeatother
% Allow footnotes in longtable head/foot
\IfFileExists{footnotehyper.sty}{\usepackage{footnotehyper}}{\usepackage{footnote}}
\makesavenoteenv{longtable}
\usepackage{graphicx}
\makeatletter
\def\maxwidth{\ifdim\Gin@nat@width>\linewidth\linewidth\else\Gin@nat@width\fi}
\def\maxheight{\ifdim\Gin@nat@height>\textheight\textheight\else\Gin@nat@height\fi}
\makeatother
% Scale images if necessary, so that they will not overflow the page
% margins by default, and it is still possible to overwrite the defaults
% using explicit options in \includegraphics[width, height, ...]{}
\setkeys{Gin}{width=\maxwidth,height=\maxheight,keepaspectratio}
% Set default figure placement to htbp
\makeatletter
\def\fps@figure{htbp}
\makeatother

\usepackage{float}
\usepackage{graphicx}
\usepackage{titlesec}
\usepackage{fancyhdr}
\usepackage{setspace}
\usepackage{tocloft}
\usepackage{booktabs}
\usepackage{longtable}
\usepackage{array}
\usepackage{multirow}
\usepackage{wrapfig}
\usepackage{colortbl}
\usepackage{pdflscape}
\usepackage{tabu}
\usepackage{threeparttable}
\usepackage{threeparttablex}
\usepackage{makecell}
\usepackage{xcolor}

% Configuración de floats
\floatplacement{table}{H}
\floatplacement{figure}{H}

% Configuración de encabezados y pies de página
\pagestyle{fancy}
\fancyhf{}
\fancyhead[LE,RO]{\thepage}
\fancyhead[LO]{\nouppercase{\leftmark}}
\fancyhead[RE]{\nouppercase{\rightmark}}
\renewcommand{\headrulewidth}{0.4pt}
\renewcommand{\footrulewidth}{0pt}
\fancypagestyle{plain}{
  \fancyhf{}
  \fancyhead[LE,RO]{\thepage}
  \renewcommand{\headrulewidth}{0pt}
}

% Configuración de títulos
\titleformat{\chapter}[display]
  {\normalfont\huge\bfseries\centering}
  {\chaptertitlename\ \thechapter}
  {20pt}
  {\Huge}

\titleformat{\section}
  {\normalfont\Large\bfseries}
  {\thesection}
  {1em}
  {}

\titleformat{\subsection}
  {\normalfont\large\bfseries}
  {\thesubsection}
  {1em}
  {}

% Espaciado de títulos
\titlespacing*{\chapter}{0pt}{50pt}{40pt}
\titlespacing*{\section}{0pt}{3.5ex plus 1ex minus .2ex}{2.3ex plus .2ex}
\titlespacing*{\subsection}{0pt}{3.25ex plus 1ex minus .2ex}{1.5ex plus .2ex}

% Configuración de índice
\renewcommand{\cftchapfont}{\bfseries}
\renewcommand{\cftsecfont}{\normalfont}
\renewcommand{\cftsubsecfont}{\normalfont}
\setlength{\cftbeforechapskip}{10pt}

% Páginas frontmatter sin numeración
\newcommand{\frontmatter}{%
  \pagenumbering{roman}%
  \setcounter{page}{1}%
}

\newcommand{\mainmatter}{%
  \clearpage
  \pagenumbering{arabic}%
  \setcounter{page}{1}%
}
\makeatletter
\@ifpackageloaded{caption}{}{\usepackage{caption}}
\AtBeginDocument{%
\ifdefined\contentsname
  \renewcommand*\contentsname{Table of contents}
\else
  \newcommand\contentsname{Table of contents}
\fi
\ifdefined\listfigurename
  \renewcommand*\listfigurename{List of Figures}
\else
  \newcommand\listfigurename{List of Figures}
\fi
\ifdefined\listtablename
  \renewcommand*\listtablename{List of Tables}
\else
  \newcommand\listtablename{List of Tables}
\fi
\ifdefined\figurename
  \renewcommand*\figurename{Figure}
\else
  \newcommand\figurename{Figure}
\fi
\ifdefined\tablename
  \renewcommand*\tablename{Table}
\else
  \newcommand\tablename{Table}
\fi
}
\@ifpackageloaded{float}{}{\usepackage{float}}
\floatstyle{ruled}
\@ifundefined{c@chapter}{\newfloat{codelisting}{h}{lop}}{\newfloat{codelisting}{h}{lop}[chapter]}
\floatname{codelisting}{Listing}
\newcommand*\listoflistings{\listof{codelisting}{List of Listings}}
\makeatother
\makeatletter
\makeatother
\makeatletter
\@ifpackageloaded{caption}{}{\usepackage{caption}}
\@ifpackageloaded{subcaption}{}{\usepackage{subcaption}}
\makeatother

\ifLuaTeX
  \usepackage{selnolig}  % disable illegal ligatures
\fi
\usepackage{bookmark}

\IfFileExists{xurl.sty}{\usepackage{xurl}}{} % add URL line breaks if available
\urlstyle{same} % disable monospaced font for URLs
\hypersetup{
  pdftitle={Título de la Tesis},
  pdfauthor={Nombre del Autor},
  colorlinks=true,
  linkcolor={blue},
  filecolor={Maroon},
  citecolor={blue},
  urlcolor={blue},
  pdfcreator={LaTeX via pandoc}}


\title{Título de la Tesis}
\usepackage{etoolbox}
\makeatletter
\providecommand{\subtitle}[1]{% add subtitle to \maketitle
  \apptocmd{\@title}{\par {\large #1 \par}}{}{}
}
\makeatother
\subtitle{Subtítulo de la Tesis}
\author{Nombre del Autor}
\date{2026-01-21}

\begin{document}
\frontmatter
\maketitle

\frontmatter
\begin{titlepage}
\centering
\vspace*{1.5cm}

% Espacio para logo (descomentar y ajustar ruta)
% \includegraphics[width=0.25\textwidth]{path/to/logo.png}
\vspace{1.5cm}

{\Huge\bfseries Título de la Tesis\par}
\vspace{0.8cm}
{\Large\itshape Subtítulo de la Tesis\par}
\vspace{2.5cm}

{\large Tesis para optar al grado de\par}
\vspace{0.5cm}
{\large Magíster en Ciencia de Datos\par}
\vspace{2.5cm}

{\large\bfseries Nombre del Autor\par}
\vspace{1.5cm}

{\large Universidad de Concepción\par}
{\large Facultad de [Nombre de la Facultad]\par}
\vspace{1.5cm}

{\large \@date\par}

\vfill

\end{titlepage}
\newpage
\thispagestyle{empty}
\vspace*{\fill}
\begin{center}
{\LARGE\bfseries DEDICATORIA}
\end{center}
\vspace{1.5cm}

% Espacio para dedicatoria personalizada
[Escribir aquí la dedicatoria]

\vspace*{\fill}
\newpage
\thispagestyle{empty}

\renewcommand*\contentsname{Table of contents}
{
\hypersetup{linkcolor=}
\setcounter{tocdepth}{2}
\tableofcontents
}
\listoffigures
\listoftables

\setstretch{1.5}
\mainmatter
\newpage
\thispagestyle{empty}
\vspace*{\fill}
\begin{center}
{\LARGE\bfseries AGRADECIMIENTOS}
\end{center}
\vspace{1.5cm}

\% Espacio para agradecimientos personalizados {[}Escribir aquí los
agradecimientos{]}

\vspace*{\fill}
\newpage
\thispagestyle{plain}

\% Índice General \tableofcontents \newpage

\% Índice de Figuras

\listoffigures
\newpage

\% Índice de Tablas

\listoftables
\newpage

\% Iniciar numeración arábiga para el contenido principal \mainmatter

\chapter{Introducción}\label{introducciuxf3n}

{[}Contenido de la introducción{]}

\chapter{Presentación de la
Propuesta}\label{presentaciuxf3n-de-la-propuesta}

\section{Problema}\label{problema}

{[}Descripción del problema{]}

\section{Acercamiento, actores,
dimensiones}\label{acercamiento-actores-dimensiones}

{[}Descripción del acercamiento, actores y dimensiones{]}

\section{Descripción de la
propuesta}\label{descripciuxf3n-de-la-propuesta}

\hyperref[descripciuxf3n-de-la-propuesta]{Descripción de la propuesta}

\section{Descripción del MVP y
objetivos}\label{descripciuxf3n-del-mvp-y-objetivos}

\hyperref[descripciuxf3n-del-mvp-y-objetivos]{Descripción del MVP y
objetivos}

\section{Descripción de
validación}\label{descripciuxf3n-de-validaciuxf3n}

\hyperref[descripciuxf3n-de-validaciuxf3n]{Descripción de validación}

\chapter{Obtención de Datos}\label{obtenciuxf3n-de-datos}

{[}Contenido sobre obtención de datos{]}

\chapter{Depuración de Datos}\label{depuraciuxf3n-de-datos}

\section{Eliminación Manual de
Columnas}\label{eliminaciuxf3n-manual-de-columnas}

{[}Contenido sobre eliminación manual de columnas{]}

\section{Tratado de Valores Nulos}\label{tratado-de-valores-nulos}

{[}Contenido sobre tratamiento de valores nulos{]}

\section{Modificación Valores
Categóricos}\label{modificaciuxf3n-valores-categuxf3ricos}

{[}Contenido sobre modificación de valores categóricos{]}

\section{Estandarización de Datos}\label{estandarizaciuxf3n-de-datos}

{[}Contenido sobre estandarización de datos{]}

\section{Eliminación de columnas con baja
variabilidad}\label{eliminaciuxf3n-de-columnas-con-baja-variabilidad}

{[}Contenido sobre eliminación de columnas con baja variabilidad{]}

\section{Eliminación de columnas con correlación
alta}\label{eliminaciuxf3n-de-columnas-con-correlaciuxf3n-alta}

{[}Contenido sobre eliminación de columnas con correlación alta{]}

\chapter{Exploración de datos}\label{exploraciuxf3n-de-datos}

{[}Contenido sobre exploración de datos{]}

\chapter{Modelado de datos}\label{modelado-de-datos}

\section{Clustering K-Means}\label{clustering-k-means}

{[}Contenido sobre clustering K-Means{]}

\section{Clustering DBSCAN}\label{clustering-dbscan}

{[}Contenido sobre clustering DBSCAN{]}

\section{Clustering Jerárquico}\label{clustering-jeruxe1rquico}

{[}Contenido sobre clustering jerárquico{]}

\section{Determinación de Modelo más
Optimo}\label{determinaciuxf3n-de-modelo-muxe1s-optimo}

{[}Contenido sobre determinación del modelo más óptimo{]}

\chapter{Interpretación de
resultados}\label{interpretaciuxf3n-de-resultados}

\section{Interpretación Componentes
Principales}\label{interpretaciuxf3n-componentes-principales}

{[}Contenido sobre interpretación de componentes principales{]}

\section{Interpretación de
Clusters}\label{interpretaciuxf3n-de-clusters}

{[}Contenido sobre interpretación de clusters{]}

\section{Validación de los
Resultados}\label{validaciuxf3n-de-los-resultados}

{[}Contenido sobre validación de resultados{]}

\chapter{Conclusión}\label{conclusiuxf3n}

{[}Contenido de la conclusión{]}

\chapter{Referencias}\label{referencias}


\backmatter


\end{document}
